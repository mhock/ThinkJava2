\chapter{Working with Strings and immutable objects}
\label{immutable}

\index{object-oriented}

Java is an ``object-oriented'' language, which means that it uses objects to represent data {\em and} provide methods related to them.
This way of organizing programs is a powerful design concept, and we will introduce it gradually throughout the remainder of the book.

This chapter introduces some of the mechanics of working with objects. It focuses on {\bf immutable} objects, which cannot change state.

\index{object}
\index{System.in}
\index{System.out}

An {\bf object} is a collection of data that provides a set of methods.
For example, \java{Scanner}, which we saw in Section~\ref{scanner}, is an object that provides methods for parsing input.
\java{System.out} and \java{System.in} are also objects.

Strings are objects, too.
They contain characters and provide methods for manipulating character data.
Other data types, like \java{Integer}, contain numbers and provide methods for manipulating number data.
We will explore some of those methods in this chapter.

\section{String iteration}

\index{iteration}

The following loop iterates the characters in \java{fruit} and displays them, one on each line:

\begin{code}
	for (int i = 0; i < fruit.length(); i++) {
		char letter = fruit.charAt(i);
		System.out.println(letter);
	}
\end{code}

\index{string!length}
\index{length!string}

Strings provide a method called \java{length} that returns the number of characters in the string.
Because it is a method, you have to invoke it with the empty argument list, \java{()}.
When \java{i} is equal to the length of the string, the condition becomes \java{false} and the loop terminates.

To find the last letter of a string, you might be tempted to do something like:

\begin{code}
	int length = fruit.length();
	char last = fruit.charAt(length);      // wrong!
\end{code}

\index{StringIndexOutOfBoundsException}
\index{exception!StringIndexOutOfBounds}

This code compiles and runs, but invoking the \java{charAt} method throws a \\ \java{StringIndexOutOfBoundsException}.
The problem is that there is no sixth letter in \java{"banana"}.
Since we started counting at 0, the 6 letters are indexed from 0 to 5.
To get the last character, you have to subtract 1 from \java{length}.

\begin{code}
	int length = fruit.length();
	char last = fruit.charAt(length - 1);  // correct
\end{code}

Many string algorithms involve reading one string and building another.
For example, to reverse a string, we can add one character at a time:

\begin{code}
	public static String reverse(String s) {
		String r = "";
		for (int i = s.length() - 1; i >= 0; i--) {
			r += s.charAt(i);
		}
		return r;
	}
\end{code}

\index{empty string}

The initial value of \java{r} is \java{""}, which is the {\bf empty string}.
The loop iterates the letters of \java{s} in reverse order.
Each time through the loop, it creates a new string and assigns it to \java{r}.
When the loop exits, \java{r} contains the letters from \java{s} in reverse order.
So the result of \java{reverse("banana")} is \java{"ananab"}.


\section{The indexOf method}

\index{indexOf}

To search for a specific character in a string, you could write a \java{for} loop and use \java{charAt} like in the previous section.
However, the \java{String} class already provides a method for doing just that.

\begin{code}
	String fruit = "banana";
	int index = fruit.indexOf('a');     // returns 1
\end{code}

This example finds the index of \java{'a'} in the string.
But the letter appears three times, so it's not obvious what \java{indexOf} should do.
According to the documentation, it returns the index of the {\em first} appearance.

To find subsequent appearances, you can use another version of \java{indexOf}, which takes a second argument that indicates where in the string to start looking.

\begin{code}
	int index = fruit.indexOf('a', 2);  // returns 3
\end{code}

To visualize how \java{indexOf} and other \java{String} methods work, it helps to draw a picture like Figure~\ref{fig.banana}.
The previous code starts at index 2 (the first \java{'n'}) and finds the next \java{'a'}, which is at index 3.

\index{memory diagram}
\index{diagram!memory}

\begin{figure}[!ht]
	\begin{center}
		\includegraphics{figs/banana.pdf}
		\caption{Memory diagram for a \java{String} of six characters.}
		\label{fig.banana}
	\end{center}
\end{figure}

%\begin{center}
%\begin{tabular}{c|c|c|c|c|c}
%%\hline
%b & a & n & a & n & a \\
%\hline
%0 & 1 & 2 & 3 & 4 & 5 \\
%%\hline
%\end{tabular}
%\end{center}

If the character happens to appear at the starting index, the starting index is the answer.
So \java{fruit.indexOf('a', 5)} returns \java{5}.
If the character does not appear in the string, \java{indexOf} returns \java{-1}.
Since indexes cannot be negative, this value indicates the character was not found.

You can also use \java{indexOf} to search for an entire string, not just a single character.
For example, the expression \java{fruit.indexOf("nan")} returns \java{2}.

\section{Counting characters}

A word is said to be a ``doubloon'' if every letter that appears in the word appears exactly twice.
Here are some example doubloons found in the dictionary:

\begin{quote}
	Abba, Anna, appall, appearer, appeases, arraigning, beriberi, bilabial, Caucasus, coco, Dada, deed, Emmett, Hannah, horseshoer, intestines, Isis, mama, Mimi, murmur, noon, Otto, papa, peep, reappear, redder, sees, Toto
\end{quote}

We will write a method called \java{isDoubloon} that takes a string and checks whether it is a doubloon.
(To ignore case, we would invoke the \java{toLowerCase} method before checking.)

One algorithm is to loop through the string 26 times, once for each lowercase letter:

\begin{code}
	for (char c = 'a'; c <= 'z'; c++) {
		// count how many times the letter appears
		// if the count is not 0 or 2, return false
	}
\end{code}

This ``nested loops'' approach is inefficient, especially when the string is long (e.g., one billion characters).
Another algorithm would initialize 26 variables to zero, loop through the string once, and use a giant \java{if} statement update the variable for each letter.
But who wants to declare 26 variables?

That's where arrays come in.
We can declare a single variable that stores 26 integers.
Rather than use an \java{if} statement to update each value, we can use arithmetic to update the $n$th value directly.

Based on the approach from Section~\ref{singlepass}, we will create an array of 26 integers to count how many times each letter appears.
We convert the string to lowercase, so that we can treat \java{'A'} and \java{'a'} (for example) as the same latter.

\begin{code}
	int[] counts = new int[26];
	String lower = s.toLowerCase();
\end{code}

We can use a \java{for} loop to iterate each character in the string.
To update the \java{counts} array, we need to compute the index that corresponds to each character.
Fortunately, Java allows you to perform arithmetic on characters.

\begin{code}
	for (int i = 0; i < lower.length(); i++) {
		char letter = lower.charAt(i);
		int index = letter - 'a';
		counts[index]++;
	}
\end{code}

\index{toCharArray}

To simplify the code, it would be nice to use an enhanced \java{for} loop.
The enhanced \java{for} loop does not work with strings directly, but you can convert any string to a character array and iterate that instead:

\begin{code}
	for (char letter : lower.toCharArray()) {
		int index = letter - 'a';
		counts[index]++;
	}
\end{code}

After counting all the characters in the \java{lower} string, we need one last \java{for} loop to determine whether each letter appears 0 or 2 times.

\begin{code}
	for (int count : counts) {
		if (count != 0 && count != 2) {
			return false;  // not a doubloon
		}
	}
	return true;  // is a doubloon
\end{code}

Like in Section~\ref{traversal}, we can return immediately if the inner condition is true (which, in this example, means that the word is not a doubloon).
If we make it all the way through the \java{for} loop, we know that all counts are 0 or 2.

\begin{code}
	public static boolean isDoubloon(String s) {
		// count the number of times each letter appears
		int[] counts = new int[26];
		String lower = s.toLowerCase();
		for (char letter : lower.toCharArray()) {
			int index = letter - 'a';
			counts[index]++;
		}
		// determine whether the given word is a doubloon
		for (int count : counts) {
			if (count != 0 && count != 2) {
				return false;
			}
		}
		return true;
	}
\end{code}

And some test code:

\begin{code}
	public static void main(String[] args) {
		System.out.println(isDoubloon("Mama"));  // true
		System.out.println(isDoubloon("Lama"));  // false
	}
\end{code}


\section{String comparison}
\label{strcmp}

\index{equals}
\index{string!comparing}

To compare two strings, it may be tempting to use the \java{==} and \java{!=} operators.

\begin{code}
	String name1 = "Alan Turing";
	String name2 = "Ada Lovelace";
	if (name1 == name2) {                 // wrong!
		System.out.println("The names are the same.");
	}
\end{code}

This code compiles and runs, and sometimes it gets the answer right.
But sometimes it gets the answer wrong.
If you give it two different strings that contain the same letters, the condition will be \java{false}.

The problem is that the \java{==} operator checks whether the two variables refer to the {\em same object} by comparing the references.
We'll learn more about references in the next chapter.
The correct way to compare strings is with the \java{equals} method, like this:

\begin{code}
	if (name1.equals(name2)) {
		System.out.println("The names are the same.");
	}
\end{code}

This example invokes \java{equals} on \java{name1} and passes \java{name2} as an argument.
The \java{equals} method returns \java{true} if the strings contain the same characters; otherwise it returns \java{false}.

\index{compareTo}

If the strings differ, we can use \java{compareTo} to see which comes first in alphabetical order:

\begin{code}
	int diff = name1.compareTo(name2);
	if (diff == 0) {
		System.out.println("The names are the same.");
	} else if (diff < 0) {
		System.out.println("name1 comes before name2.");
	} else if (diff > 0) {
		System.out.println("name2 comes before name1.");
	}
\end{code}

The return value from \java{compareTo} is the difference between the first characters in the strings that are not the same.
In the preceding code, \java{compareTo} returns positive 8, because the second letter of \java{"Ada"} comes before the second letter of \java{"Alan"} by 8 letters.

If the strings are equal, their difference is zero.
If the first string (the one on which the method is invoked) comes first in the alphabet, the difference is negative.
Otherwise, the difference is positive.

\index{case-sensitive}

Both \java{equals} and \java{compareTo} are case-sensitive.
In Unicode, uppercase letters come before lowercase letters.
So \java{"Ada"} comes before \java{"ada"}.


\section{Substrings}

\index{substring}

The \java{substring} method returns a new string that copies letters from an existing string, starting at the given index.

\begin{itemize}
	\item \java{fruit.substring(0)} returns \java{"banana"}
	\item \java{fruit.substring(2)} returns \java{"nana"}
	\item \java{fruit.substring(6)} returns \java{""}
\end{itemize}

The first example returns a copy of the entire string.
The second example returns all but the first two characters.
As the last example shows, \java{substring} returns the empty string if the argument is the length of the string.

Like most string methods, \java{substring} is overloaded.
That is, there are other versions of \java{substring} that have different parameters.
If it's invoked with two arguments, they are treated as a start and end index:

\begin{itemize}
	\item \java{fruit.substring(0, 3)} returns \java{"ban"}
	\item \java{fruit.substring(2, 5)} returns \java{"nan"}
	\item \java{fruit.substring(6, 6)} returns \java{""}
\end{itemize}

Notice that the character indicated by the end index is {\em not} included.
Defining \java{substring} this way simplifies some common operations.
For example, to select a substring with length \java{len}, starting at index \java{i}, you could write \java{fruit.substring(i, i + len)}.
%So \java{fruit.substring(2, 2 + 3)} returns \java{"nan"}.


\section{String formatting}

\index{printf}

In Section~\ref{printf}, we learned how to use \java{System.out.printf} to display formatted output.
Sometimes programs need to create strings that are formatted a certain way, but not display them immediately, or ever.
For example, the following method returns a time string in 12-hour format:

\begin{code}
	public static String timeString(int hour, int minute) {
		String ampm;
		if (hour < 12) {
			ampm = "AM";
			if (hour == 0) {
				hour = 12;  // midnight
			}
		} else {
			ampm = "PM";
			hour = hour - 12;
		}
		return String.format("%02d:%02d %s", hour, minute, ampm);
	}
\end{code}

\index{string!format}

\java{String.format} takes the same arguments as \java{System.out.printf}: a format specifier followed by a sequence of values.
The main difference is that \java{System.out.printf} displays the result on the screen.
\java{String.format} creates a new string, but does not display anything.

In this example, the format specifier \java{\%02d} means ``two digit integer padded with zeros'', so \java{timeString(19, 5)} returns the string \java{"07:05 PM"}.
As an exercise, try writing two nested \java{for} loops (in \java{main}) that invoke \java{timeString} and display all possible times over a 24-hour period.

At some point today, skim through the documentation for \java{String}.
Knowing what other methods are there will help you avoid reinventing the wheel.
The easiest way to find documentation for Java classes is to do a web search for ``Java'' and the name of the class.


\section{Primitives vs objects}

\index{primitive}

Not everything in Java is an object: \java{int}, \java{double}, \java{char}, and \java{boolean} are examples of {\bf primitive} types.
%We will explain some of the differences between object types and primitive types as we go along.
When you declare a variable with a primitive type, Java reserves a small amount of memory to store its value.
Figure~\ref{fig.mem1} shows how the following values are stored memory.

\begin{code}
int number = -2;
char symbol = '!';
\end{code}

\begin{figure}[!ht]
\begin{center}
\includegraphics[scale=0.85]{figs/mem1.pdf}
\caption{Memory diagram of two primitive variables.}
\label{fig.mem1}
\end{center}
\end{figure}

\index{memory diagram}
\index{diagram!memory}

As we learned in Section~\ref{elements}, an array variable stores a {\em reference} to an array.
That's because the array itself is too large to fit in the variable's memory.
For example, \java{char[] array = \{'c', 'a', 't'\};} contains three characters.

\begin{figure}[!ht]
\begin{center}
\includegraphics[scale=0.85]{figs/mem2.pdf}
\caption{Memory diagram of an array of characters.}
\label{fig.mem2}
\end{center}
\end{figure}

When drawing memory diagrams, we use an arrow to represent the location of the array, as in Figure~\ref{fig.mem2}.
The actual memory location (the {\em value} of the array variable) is an integer chosen by Java at run-time.

Objects work in a similar way.
When you declare an object variable, it will store a reference to an object.
In contrast to arrays, which store multiple elements of the same data type, objects can be used to {\bf encapsulate} any type of data.
%We will learn more about this concept in Chapter~\ref{mutable}.

For example, a \java{String} object encapsulates a character array.
Figure~\ref{fig.mem3} illustrates how strings are stored in memory.

\begin{figure}[!ht]
\begin{center}
\includegraphics[scale=0.85]{figs/mem3.pdf}
\caption{Memory diagram of a \java{String} object.}
\label{fig.mem3}
\end{center}
\end{figure}

Behind the scenes, the code \java{String word = "dog";} creates an array of the characters \java{'d'}, \java{'o'}, and \java{'g'}, and stores the reference to that array in a \java{String} object.
The variable \java{word} contains a reference to the \java{String} object.

\index{string!comparing}

To test whether two integers (or other primitive types) are equal, you simply use the \java{==} operator.
But as we learned in Section~\ref{strcmp}, you need to use the \java{equals} method to compare strings.
The \java{equals} method traverses the arrays and tests whether they contain the same characters.

On the other hand, two \java{String} objects with the same characters would not be considered equal in the \java{==} sense.
The \java{==} operator, when applied to string variables, only tests whether they refer to the {\em same} object.


%\section{The null keyword}

\index{null}

In Java, the keyword \java{null} is a special value that means ``no object''.
You can initialize object and array variables this way:

\begin{code}
String name = null;
int[] combo = null;
\end{code}

The value \java{null} is represented in memory diagrams by a small box with no arrow, as in Figure~\ref{fig.mem4}.
In other words, the variables do not reference anything.

\begin{figure}[!ht]
\begin{center}
\includegraphics[scale=0.85]{figs/mem4.pdf}
\caption{Memory diagram showing variables that are \java{null}.}
\label{fig.mem4}
\end{center}
\end{figure}

\index{NullPointerException}
\index{exception!NullPointer}

If you try to use a variable that is \java{null} by invoking a method or accessing an element, Java throws a \java{NullPointerException}.

\begin{code}
System.out.println(name.length());  // NullPointerException
System.out.println(combo[0]);       // NullPointerException
\end{code}

On the other hand, it is perfectly fine to pass a \java{null} reference as an argument to a method, or to receive one as a return value.
In these situations, \java{null} is often used to represent a special condition or indicate an error.


\section{Strings are immutable}

If the Java library didn't have a \java{String} class, we would have to use character arrays to store and manipulate text.
Operations like concatenation (\java{+}), \java{indexOf}, and \java{substring} would be difficult and inconvenient.
Fortunately, Java does have a \java{String} class that provides these and other methods.

\index{toUpperCase}
\index{toLowerCase}
\index{immutable}

For example, the methods \java{toLowerCase} and \java{toUpperCase} convert uppercase letters to lowercase, and vice versa.
These methods are often a source of confusion, because it sounds like they modify strings.
But neither these methods nor any others can change a string, because strings are {\bf immutable}.

Believe it or not, immutability is already a familiar concept to you. Consider the number 5. You may use this number to describe a quantity that can change, such as the number of words in an editable document. But it doesn't make sense to be able change the number 5 itself to mean ``six things''. Like the childhood game of ``opposite day'' where every word now means its opposite, changing an immutable constant like 5 would break many assumptions and drive everyone crazy!

Instead of changing the string, when you invoke \java{toUpperCase} on a string, you get another \java{String} object as a result.
For example:

\begin{code}
String name = "Alan Turing";
String upperName = name.toUpperCase();
\end{code}

%\index{Turing, Alan}

After these statements run, \java{upperName} refers to the string \java{"ALAN TURING"}.
But \java{name} still refers to \java{"Alan Turing"}.
A common mistake is to assume that \java{toUpperCase} somehow affects the original string:

\begin{code}
String name = "Alan Turing";
name.toUpperCase();           // ignores the return value
System.out.println(name);
\end{code}

The previous code displays \java{"Alan Turing"}, because the value of \java{name} (i.e., the reference to the original \java{String} object) never changes.
If you want to change \java{name} to be uppercase, then you need to assign the return value to it:

\begin{code}
String name = "Alan Turing";
name = name.toUpperCase();    // references the new string
System.out.println(name);
\end{code}

Here is one more demonstration of references and immutability. Consider the following class:

\begin{code}
public class ImmutableDemo {
	public static void makeUpperCase(String name) {
		name = name.toUpperCase();
	}
	
	public static void main(String[] args) {
		String name = "Alan Turing";
		makeUpperCase(name);
		System.out.println(name); // What does this print?
	}
}
\end{code}

When the above \java{main} method is run, a reference to the \java{name} variable is passed to the \java{makeUpperCase} method. That method has its own parameter variable named \java{name}. Both variables refer to the same underlying String object. But when the \java{makeUpperCase} method's \java{name} variable is modified, it does not modify the \java{main} method's \java{name} variable. And as we know, the \java{toUpperCase} method does not modify the underlying String object because Strings are immutable.

Stay tuned! In section~\ref{args}, we'll finally learn about the \java{args} parameter.


\index{replace}

A similar method is \java{replace}, which finds and replaces instances of one string within another.
This example replaces \java{"Computer Science"} with \java{"CS"}:

\begin{code}
String text = "Computer Science is fun!";
text = text.replace("Computer Science", "CS");
\end{code}

%This example demonstrates a common way to work with string methods.
%It invokes \java{text.replace}, which returns a reference to a new string, \java{"CS is fun!"}.
%Then it assigns the new string to \java{text}, replacing the old string.

As with \java{toUpperCase}, assigning the return value (to \java{text}) is important.
If you don't assign the return value, invoking \java{text.replace} has no effect.

% ABD: Too many new ideas here: the most important one is that you have to do something with the return value. It's not a good time to appreciate the glory of immutability.
% CSM: Now that strings were introduced previously, I would like to make this chapter say more about immutability. But we won't get to the full glory until the next chapter.

\subsection{Why are Strings immutable?}
In some programming languages, strings can be modified. But by making \java{String} immutable in Java, the same object can be reused to represent the same information, saving memory. In addition, when we learn about creating our own classes, we will see that making \java{String} immutable aids in the protection of data integrity and reliability. If someone else gives your code a String object, you can use that String object with certainty that the other code will not change it while you are using it!


\section{Wrapper classes}

Primitive values (like \java{int}s, \java{double}s, and \java{char}s) cannot be \java{null}, and they do not provide methods.
For example, you can't invoke \java{equals} on an \java{int}:

\begin{code}
int i = 5;
System.out.println(i.equals(5));  // compiler error
\end{code}

\index{wrapper class}
\index{class!wrapper}
\index{Character}
\index{Integer}
\index{Double}

But for each primitive type, there is a corresponding {\bf wrapper class} in the Java library.
The wrapper class for \java{int} is named \java{Integer}, with a capital \java{I}.

\begin{code}
Integer i = new Integer(5);
System.out.println(i.equals(5));  // displays true
\end{code}

Other wrapper classes include \java{Boolean}, \java{Character}, \java{Double}, and \java{Long}.
They are in the \java{java.lang} package, so you can use them without importing them.

Like strings, objects from wrapper classes are immutable.
And you need to use the \java{equals} method to compare them.

\begin{code}
Integer x = new Integer(123);
Integer y = new Integer(123);
if (x == y) {                           // false
    System.out.println("x and y are the same object");
}
if (x.equals(y)) {                      // true
    System.out.println("x and y have the same value");
}
\end{code}

Because \java{x} and \java{y} refer to different \java{Integer} objects, the code only displays ``x and y have the same value''.

Each wrapper class defines the constants \java{MIN_VALUE} and \java{MAX_VALUE}.
For example, \java{Integer.MIN_VALUE} is \java{-2147483648}, and \java{Integer.MAX_VALUE} is \java{2147483647}.
Because these constants are available in wrapper classes, you don't have to remember them, and you don't have to write them yourself.

\index{parse}

Wrapper classes also provide methods for converting strings to and from primitive types.
For example, \java{Integer.parseInt} converts a string to (you guessed it) an integer.
In this context, {\bf parse} means ``read and translate''.

\begin{code}
String str = "12345";
int num = Integer.parseInt(str);
\end{code}

The other wrapper classes provide similar methods, like \java{Double.parseDouble} and \java{Boolean.parseBoolean}.
They also each provide \java{toString}, which returns a string representation of a value:

\begin{code}
int num = 12345;
String str = Integer.toString(num);
\end{code}

The result is the string \java{"12345"}, which as you now understand, is stored internally in a character array \java{\{'1', '2', '3', '4', '5'\}}.

\index{NumberFormatException}
\index{exception!NumberFormat}

It's always possible to convert a primitive value to a string, but not the other way around.
The following code throws a \java{NumberFormatException}.

\begin{code}
String str = "five";
int num = Integer.parseInt(str);  // NumberFormatException
\end{code}


\section{Command-line arguments}
\label{args}
\index{args}
\index{command-line interface}

Now that you know about strings, arrays, and wrapper classes, we can {\em finally} explain the \java{args} parameter of the \java{main} method, which we have been ignoring since Chapter~\ref{theway}.
If you are unfamiliar with the command-line interface, please read Appendix~\ref{commandline}.

Let's write a program to find the maximum value in a sequence of numbers.
Rather than read the numbers from \java{System.in} using a \java{Scanner}, we'll pass them as command-line arguments.
Here is a starting point:

\begin{code}
public class Max {
    public static void main(String[] args) {
        System.out.println(Arrays.toString(args));
    }
}
\end{code}

You can run this program from the command line by typing:

\begin{stdout}
java Max
\end{stdout}

\index{empty array}

The output indicates that \java{args} is an {\bf empty array}; that is, it has no elements:

\begin{stdout}
[]
\end{stdout}

If you provide additional values on the command line, they are passed as arguments to \java{main}.
For example, if you run the program like this:

\begin{stdout}
java Max 10 -3 55 0 14
\end{stdout}

The output is:

\begin{stdout}
[10, -3, 55, 0, 14]
\end{stdout}

It's not clear from the output, but the elements of \java{args} are strings.
So \java{args} is the array \java{\{"10", "-3", "55", "0", "14"\}}.
To find the maximum number, we have to convert the arguments to integers.

The following code uses an enhanced \java{for} loop to parse the arguments (using the \java{Integer} wrapper class) and find the largest value:

\begin{code}
int max = Integer.MIN_VALUE;
for (String arg : args) {
    int value = Integer.parseInt(arg);
    if (value > max) {
        max = value;
    }
}
System.out.println("The max is " + max);
\end{code}

We begin by initializing \java{max} to the smallest (most negative) number an \java{int} can represent.
That way, the first value we parse will replace \java{max}.
As we find larger values, they will replace \java{max} as well.

If \java{args} is empty, the result will be \java{MIN_VALUE}.
We can prevent this situation from happening by checking \java{args} at the beginning of the program:

\begin{code}
if (args.length == 0) {
    System.err.println("Usage: java Max <numbers>");
    return;
}
\end{code}

It's customary for programs that require command-line arguments to display a ``usage'' message when there are no arguments given.
For example, if you run {\tt javac} or {\tt java} from the command line without any arguments, you will get a very long message.


\section{BigInteger arithmetic}
% CSM based on text from V6 Exercise 10.4

It might not be clear at this point why you would ever need an integer object when you can just use an \java{int} or \java{long}.
One advantage is the variety of methods that \java{Integer} and \java{Long} provide.
But there is another reason: when you need very large integers that exceed \java{Long.MAX_VALUE}.

\index{BigInteger}

\java{BigInteger} is a Java class that can represent arbitrarily large integers.
There is no upper bound except the limitations of memory size and processing speed.
Take a minute to read the documentation, which you can find by doing a web search for ``Java BigInteger''.

\index{java.math}

To use BigIntegers, you have to \java{import java.math.BigInteger} at the beginning of your program.
There are several ways to create a BigInteger, but the simplest uses \java{valueOf}.
The following code converts a \java{long} to a BigInteger:

\begin{code}
long x = 17;
BigInteger big = BigInteger.valueOf(x);
\end{code}

You can also create BigIntegers from strings.
For example, here is a 20-digit integer that is too big to store using a \java{long}.

\begin{code}
String s = "12345678901234567890";
BigInteger bigger = new BigInteger(s);
\end{code}

Notice the difference in the previous two examples: you use \java{valueOf} to convert integers, and \java{new BigInteger} to convert strings.

Since BigIntegers are not primitive types, the usual math operators don't work.
Instead, we have to use methods like \java{add}.
To add two BigIntegers, we invoke \java{add} on one and pass the other as an argument.

\begin{code}
BigInteger a = BigInteger.valueOf(17);
BigInteger b = BigInteger.valueOf(1700000000);
BigInteger c = a.add(b);
\end{code}

Like strings, \java{BigInteger} objects are immutable.
Methods like \java{add}, \java{multiply}, and \java{pow} all return new BigIntegers, rather than modify an existing one.

Internally, a BigInteger encapsulates an array of \java{int}s, similar to the way a string encapsulates an array of \java{char}s.
Each \java{int} in the array stores a portion of the BigInteger.
The methods of \java{BigInteger} traverse this array to perform addition, multiplication, etc.

For very long floating-point values, take a look at \java{java.math.BigDecimal}.
Interestingly, \java{BigDecimal} objects represent floating-point numbers internally by encapsulating a \java{BigInteger}!


\section{Program development}
\label{encapsulation}

This chapter introduces two main concepts: objects encapsulate other types of data, and they can be designed to be immutable.
Applying these concepts helps us to manage the complexity of programs as they become large.

\index{design process}
\index{encapsulation!and generalization}

Unfortunately, computer science has a lot of overloaded terms.
Another use of the term ``encapsulation'' applies to methods.
In this section, we present a {\bf design process} called ``encapsulation and generalization''.

One challenge of programming, especially for beginners, is figuring out how to divide up a program into methods.
The process of encapsulation and generalization allows you to design as you go along.
The steps are:

\begin{enumerate}
\item Write a few lines of code in \java{main} or another method, and test them.
\item When they are working, wrap them in a new method, and test again.
\item If it's appropriate, replace literal values with variables and parameters.
\end{enumerate}

Encapsulation and generalization is similar to ``incremental development'' (see Section~\ref{distance}), in the sense that you write a little code, test it, and repeat.
But you don't need to begin with an exact method definition and stub.

\index{table!two-dimensional}

To demonstrate this process, we'll develop methods that display multiplication tables.
Here is a loop that displays the multiples of two, all on one line:

\begin{code}
int i = 1;
while (i <= 6) {
    System.out.printf("%4d", 2 * i);
    i = i + 1;
}
System.out.println();
\end{code}

\index{loop variable}
\index{variable!loop}

The first line initializes a variable named \java{i}, which is going to act as the loop variable.
As the loop executes, the value of \java{i} increases from 1 to 6; when \java{i} is 7, the loop terminates.

Each time through the loop, we display the value \java{2 * i} padded with spaces so it's four characters wide.
Since we use \java{System.out.printf}, the output appears on a single line.

After the loop, we call \java{println} to print a newline and complete the line.
Remember that in some environments, none of the output is displayed until the line is complete.

The output of the code so far is:

\begin{stdout}
   2   4   6   8  10  12
\end{stdout}

\index{encapsulate}

The next step is to {\bf encapsulate} or wrap this code in a method.
Here's what it looks like:

\begin{code}
public static void printRow() {
    int i = 1;
    while (i <= 6) {
        System.out.printf("%4d", 2 * i);
        i = i + 1;
    }
    System.out.println();
}
\end{code}

\index{generalize}

Next, we {\bf generalize} the method by replacing the constant value, \java{2}, with a parameter, \java{n}.
This step is called ``generalization'' because it makes the method more general (less specific).

\begin{code}
public static void printRow(int n) {
    int i = 1;
    while (i <= 6) {
        System.out.printf("%4d", n * i);  // generalized n
        i = i + 1;
    }
    System.out.println();
}
\end{code}

Invoking this method with the argument 2 yields the same output as before.
With the argument 3, the output is:

\begin{stdout}
   3   6   9  12  15  18
\end{stdout}

%And with argument 4, the output is:
%
%\begin{stdout}
%   4   8  12  16  20  24
%\end{stdout}

By now you can probably guess how we are going to display a multiplication table: we'll invoke \java{printRow} repeatedly with different arguments.
In fact, we'll use another loop to iterate through the rows.

\begin{code}
int i = 1;
while (i <= 6) {
    printRow(i);
    i = i + 1;
}
\end{code}

And the output looks like this:

\begin{stdout}
   1   2   3   4   5   6
   2   4   6   8  10  12
   3   6   9  12  15  18
   4   8  12  16  20  24
   5  10  15  20  25  30
   6  12  18  24  30  36
\end{stdout}

%The format specifier \java{\%4d} in \java{printRow} causes the output to align vertically, regardless of whether the numbers are one or two digits.


\section{More generalization}

The previous result is similar to the ``nested loops'' approach in Section~\ref{nested}.
However, the inner loop is now encapsulated in the \java{printRow} method.
We can encapsulate the outer loop in a method too:

\begin{code}
public static void printTable() {
    int i = 1;
    while (i <= 6) {
        printRow(i);
        i = i + 1;
    }
}
\end{code}

The initial version of \java{printTable} always displays six rows.
We can generalize it by replacing the literal \java{6} with a parameter:

\begin{code}
public static void printTable(int rows) {
    int i = 1;
    while (i <= rows) {  // generalized rows
        printRow(i);
        i = i + 1;
    }
}
\end{code}

Here is the output of \java{printTable(7)}:

\begin{stdout}
   1   2   3   4   5   6
   2   4   6   8  10  12
   3   6   9  12  15  18
   4   8  12  16  20  24
   5  10  15  20  25  30
   6  12  18  24  30  36
   7  14  21  28  35  42
\end{stdout}

That's better, but it still has a problem: it always displays the same number of columns.
We can generalize more by adding a parameter to \java{printRow}:

\begin{code}
public static void printRow(int n, int cols) {
    int i = 1;
    while (i <= cols) {  // generalized cols
        System.out.printf("%4d", n * i);
        i = i + 1;
    }
    System.out.println();
}
\end{code}

Now \java{printRow} takes two parameters: \java{n} is the value whose multiples should be displayed, and \java{cols} is the number of columns.
Since we added a parameter to \java{printRow}, we also have to change the line in \java{printTable} where it is invoked:

\begin{code}
public static void printTable(int rows) {
    int i = 1;
    while (i <= rows) {
        printRow(i, rows);  // added rows argument
        i = i + 1;
    }
}
\end{code}

When this line executes, it evaluates \java{rows} and passes the value, which is 7 in this example, as an argument.
In \java{printRow}, this value is assigned to \java{cols}.
As a result, the number of columns equals the number of rows, so we get a square 7x7 table (instead of the previous 7x6 table):

%\begin{stdout}
%   1   2   3   4   5   6   7
%   2   4   6   8  10  12  14
%   3   6   9  12  15  18  21
%   4   8  12  16  20  24  28
%   5  10  15  20  25  30  35
%   6  12  18  24  30  36  42
%   7  14  21  28  35  42  49
%\end{stdout}

When you generalize a method appropriately, you often find that it has capabilities you did not plan.
For example, you might notice that the multiplication table is symmetric.
Since $ab = ba$, all the entries in the table appear twice.
You could save ink by printing half of the table, and you would only have to change {\em one line} of \java{printTable}:

\begin{code}
printRow(i, i);  // using i for both n and cols
\end{code}

In English, the length of each row is the same as its row number.
The result is a triangular multiplication table.

\begin{stdout}
   1
   2   4
   3   6   9
   4   8  12  16
   5  10  15  20  25
   6  12  18  24  30  36
   7  14  21  28  35  42  49
\end{stdout}

Generalization makes code more versatile, more likely to be reused, and sometimes easier to write.
In this example, we started with a simple idea and ended with two general-purpose methods.

%Even though the second parameter in \java{printRow} is named \java{size} and we have a variable with the same name, we can still use any value or expression we want for the argument.

%Remember, you do not pass {\em variables} to methods; you pass their current {\em values}.
%In this last example, the value of \java{i} in \java{printTable} is assigned to both \java{n} and \java{cols} in \java{printRow}.


\section{Vocabulary}

\begin{description}

\term{object}
A collection of related data that comes with a set of methods that operate on the data.

\term{primitive}
A data type that stores a single value and provides no methods.

\term{immutable}
An object that, once created, cannot be modified.
Strings are immutable by design.

\term{wrapper class}
Classes in \java{java.lang} that provide constants and methods for working with primitive types.

\term{parse}
To read a string and interpret or translate it.

\term{empty array}
An array with no elements and a length of zero.

\term{design process}
A process for determining what methods a class or program should have.
%So far we have seen ``incremental development'' and ``encapsulation and generalization''.

\term{encapsulate}
To wrap data inside of an object, or to wrap statements inside of a method.

\term{generalize}
To replace something unnecessarily specific (like a constant value) with something appropriately general (like a variable or parameter).

\end{description}


\section{Exercises}

The code for this chapter is in the {\tt ch09} directory of {\tt ThinkJavaCode2}.
See page~\pageref{code} for instructions on how to download the repository.
Before you start the exercises, we recommend that you compile and run the examples.


\begin{exercise}  %%V6 Ex9.1

The point of this exercise is to explore Java types and fill in some of the details that aren't covered in the chapter.

\index{concatenate}

\begin{enumerate}

\item Create a new program named {\tt Test.java} and write a \java{main} method that contains expressions that combine various types using the \java{+} operator.
For example, what happens when you ``add'' a \java{String} and a \java{char}?
Does it perform character addition or string concatenation?
What is the type of the result?
(How can you determine the type of the result?)

\item Make a bigger copy of the following table and fill it in.
At the intersection of each pair of types, you should indicate whether it is legal to use the \java{+} operator with these types, what operation is performed (addition or concatenation), and what the type of the result is.

\begin{center}
\begin{tabular}{|l|l|l|l|l|l|} \hline
        &  boolean  &  ~char~  &  ~~int~~  &  double  &  String \\ \hline
boolean &           &          &           &          &         \\ \hline
char    &           &          &           &          &         \\ \hline
int     &           &          &           &          &         \\ \hline
double  &           &          &           &          &         \\ \hline
String  &           &          &           &          &         \\ \hline
\end{tabular}
\end{center}

\item Think about some of the choices the designers of Java made, based on this table.
How many of the entries seem unavoidable, as if there was no other choice?
How many seem like arbitrary choices from several equally reasonable possibilities?
Which entries seem most problematic?

\item Here's a puzzler: normally, the statement \java{x++} is exactly equivalent to \java{x = x + 1}.
But if \java{x} is a \java{char}, it's not exactly the same!
In that case, \java{x++} is legal, but \java{x = x + 1} causes an error.
Try it out and see what the error message is, then see if you can figure out what is going on.

\item What happens when you add \java{""} (the empty string) to the other types, for example, \java{"" + 5}?

%\item For each data type, what types of values can you assign to it?
%For example, you can assign an \java{int} to a \java{double} but not vice versa.

\end{enumerate}

\end{exercise}


\begin{exercise}  %%V6 Ex8.1

The goal of this exercise is to practice encapsulation and generalization using some of the examples in previous chapters.

\begin{enumerate}

\item Starting with the code in Section~\ref{traversal}, write a method called \java{powArray} that takes a \java{double} array, \java{a}, and returns a new array that contains the elements of \java{a} squared.
Generalize it to take a second argument and raise the elements of \java{a} to the given power.

\item Starting with the code in Section~\ref{enhanced}, write a method called \java{histogram} that takes an \java{int} array of scores from 0 to (but not including) 100, and returns a histogram of 100 counters.
Generalize it to take the number of counters as an argument.

\end{enumerate}

\end{exercise}


\begin{exercise}  %%V6 Ex10.4

\index{factorial}

You might be sick of the factorial method by now, but we're going to do one more version.

\begin{enumerate}

\item Create a new program called {\tt Big.java} and write an iterative version of \java{factorial} (using a \java{for} loop).

\item Display a table of the integers from 0 to 30 along with their factorials.
At some point around 15, you will probably see that the answers are not correct anymore.
Why not?

\item Convert \java{factorial} so that it performs its calculation using BigIntegers and returns a \java{BigInteger} as a result.
You can leave the parameter alone; it will still be an integer.

\item Try displaying the table again with your modified factorial method.
Is it correct up to 30?
How high can you make it go?

\end{enumerate}

\end{exercise}


\begin{exercise}  %%V6 Ex10.5

Many encryption algorithms depend on the ability to raise large integers to a power.
Here is a method that implements an efficient algorithm for integer exponentiation:

\begin{code}
public static int pow(int x, int n) {
    if (n == 0) return 1;

    // find x to the n/2 recursively
    int t = pow(x, n / 2);

    // if n is even, the result is t squared
    // if n is odd, the result is t squared times x
    if (n % 2 == 0) {
        return t * t;
    } else {
        return t * t * x;
    }
}
\end{code}

The problem with this method is that it only works if the result is small enough to be represented by an \java{int}.
Rewrite it so that the result is a \java{BigInteger}.
The parameters should still be integers, though.

You should use the \java{BigInteger} methods \java{add} and \java{multiply}.
But don't use \java{BigInteger.pow}; that would spoil the fun.

\end{exercise}


\begin{exercise}  %%V6 Ex7.5

%The purpose of this exercise is to practice using \java{BigInteger} and \java{BigDecimal}.

One way to calculate $e^x$ is to use the following infinite series expansion.
The $i$th term in the series is $x^i / i!$.
%
\[ e^x = 1 + x + x^2 / 2! + x^3 / 3! + x^4 / 4! + \ldots \]
%
\begin{enumerate}

\item Write a method called \java{myexp} that takes \java{x} and \java{n} as parameters and estimates $e^x$ by adding the first \java{n} terms of this series.
You can use the \java{factorial} method from Section~\ref{factorial} or your iterative version from the previous exercise.

\index{efficiency}

\item You can make this method more efficient by observing that the numerator of each term is the same as its predecessor multiplied by \java{x}, and the denominator is the same as its predecessor multiplied by \java{i}.

Use this observation to eliminate the use of \java{Math.pow} and \java{factorial}, and check that you get the same result.

\item Write a method called \java{check} that takes a parameter, \java{x}, and displays \java{x}, \java{myexp(x)}, and \java{Math.exp(x)}.
The output should look something like:

\begin{stdout}
1.0     2.708333333333333     2.718281828459045
\end{stdout}

You can use the escape sequence \java{"\\t"} to put a tab character between columns of a table.

\item Vary the number of terms in the series (the second argument that \java{check} sends to \java{myexp}) and see the effect on the accuracy of the result.
Adjust this value until the estimated value agrees with the correct answer when \java{x} is 1.

\item Write a loop in \java{main} that invokes \java{check} with the values 0.1, 1.0, 10.0, and 100.0.
How does the accuracy of the result vary as \java{x} varies?
Compare the number of digits of agreement rather than the difference between the actual and estimated values.

\item Add a loop in \java{main} that checks \java{myexp} with the values -0.1, -1.0, -10.0, and -100.0.
Comment on the accuracy.

\end{enumerate}

\end{exercise}


\begin{exercise}  %%V6 Ex9.3

\index{encapsulation}
\index{generalization}

%The purpose of this exercise is to review encapsulation and generalization (see Section~\ref{encapsulation}).
The following code fragment traverses a string and checks whether it has the same number of open and close parentheses:

\begin{code}
String s = "((3 + 7) * 2)";
int count = 0;

for (int i = 0; i < s.length(); i++) {
    char c = s.charAt(i);
    if (c == '(') {
        count++;
    } else if (c == ')') {
        count--;
    }
}

System.out.println(count);
\end{code}

\begin{enumerate}

\item Encapsulate this fragment in a method that takes a string argument and returns the final value of \java{count}.

\item Test your method with multiple strings, including some that are balanced and some that are not.

\item Generalize the code so that it works on any string. What could you do to generalize it more?

\end{enumerate}

\end{exercise}
